

\chapter{Einleitung und Problemhintergrund}\label{cha:introduction}

Diese Kapitel sollte die folgenden 4 Themenbereiche behandeln.

\begin{enumerate}
    \item Absatz (Universum / Forschungsfeld): Im ersten Absatz soll das Univsersum / Forschungsfeld beschrieben werden, in dem sich die Arbeit befindet. Der Leser / die Leserin soll aus dem "großen Ganzen" an die eigentliche Problemstellung herangeführt werden und wissen, in welchem Themenfeld sich diese befindet.

    \item Absatz (Problemstellung im Universum / Forschungsfeld): Im zweiten Teil wird dem Leser / der Leserin das Problem, welche im beschriebenen Universum / Forschungsfeld besteht, aufge\-zeigt bzw. beschrieben. Die Problemstellung gilt dabei als die Grundlage für die Arbeit, die hier beschrieben wird und die Methodik "WIE man es lösen würde" erklärt wird.

    \item Absatz (Lösungsansatz, um die Problemstellung zu bearbeiten): Im 3. Teil soll eine Idee bzw. ein möglicher Lösungsansatz "angeteasert" werden, mit dem man die beschriebene Problemstellung bearbeiten möchte. Dabei ist es wichtig, dass der beschriebene Lösungsweg realistisch bzw. umsetzbar und auch nachvollziehbar ist. Es soll ein möglicher Weg sein, den man gehen kann, um die Problemstellung zu bearbieten oder sich zumindest einer Lösung anzunähern. Der Leser bzw. die Leserin soll den Eindruck bekommen, dass die Problemstellung tatsächlich damit lösbar sei.

    \item Absatz (Aufbau der Arbeit): Jedes Paper hat zum Schluss einen Absatz, der beschreibt, wie die vorliegende Arbeit aufgebaut ist. Hier ein Beispiel aus einem Position Paper, das auf Englisch geschrieben wurde:
          The remainder of this paper is organised as follows: Section II summarises the related work in the field. Next, in Section III, we present the Security Evaluation Framework and explain how it can be used to evaluate the security of SDN-components. Furthermore, we show the general applicability of the proposed framework in an experimental study in Section IV. Finally, in Section V we conclude our work and give an outline of future work in the field.
\end{enumerate}

