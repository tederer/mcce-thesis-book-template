

\chapter{Stand des Wissens / Stand der Technik}

Ziel dieses Kapitels ist es, dem Leser zu zeigen, welche anderen verwandten Arbeiten im selben Universum / Forschungsfeld bereits Ergebnisse geliefert haben. Hier geht es darum, dass man zeigt, dass man das eigene Forschungsfeld kennt und die wichtigsten Arbeiten darin kurz zusammengefasst beschreibt. Genauso ist es wichtig, dass man folgenden Unterschied bzw. das DELTA herausarbeitet:

\begin{itemize}
	\item Your Work vs. Related Work
	\item ODER Was haben die anderen Arbeiten eben NICHT gemacht und darum möchte ich es in dieser Arbeir motivieren
\end{itemize}

\noindent
Dieses Kapitel ist sehr wichtig, weil es aus der Einleitung und der Problembeschreibung heraus nochmal zeigt, dass es einen NEED gibt, um dieses Problem zu lösen (weil es bis jetzt kein anderer getan hat). Folgende Key-Points gibt es bei diesem Kapitel zu beachten:

\begin{itemize}
	\item wissenschaftliche Quellen (andere Papers oder facheinschlägige Bücher) sollen identifiziert und zusammengefasst dargestellt / beschrieben werden. Dabei sollen diese wissenschaftlichen Quellen zitiert werden
	\item die wissenschaftlichen Quellen sollen sich zu den nicht wissenschaftlichen Quellen die Waage halten im
	      Verhältnis von 2/3 wissenschaftliche Quellen zu 1/3 andere Quellen
\end{itemize}

