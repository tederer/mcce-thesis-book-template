

\chapter{Wissenschaftliche Fragestellung}\label{cha:researchQuestion}

Wie in \textcite{solis2023} beschrieben, ist die Forschungsfrage  die Grundlage einer wissenschaftlichen Arbeit. Es ist deine Aufgabe, im Laufe deiner Arbeit die gestellte Forschungsfrage zu beantworten. Daher richtest du deine komplette Arbeit auf die Beantwortung deiner Forschungsfrage hin aus. Folglich wird die Forschungsfrage schon am Anfang der Arbeitsphase formuliert und sollte die folgenden Kriterien erfüllen:

\begin{itemize}
    \item präzise formuliert sein.
    \item auf ein einzelnes spezifisches Thema begrenzt sein.
    \item relevant für dein Studienfach sein.
    \item erforschbar sein.
    \item innerhalb des Zeitrahmens und Umfangs deiner Arbeit beantwortbar sein.
    \item so komplex sein, dass eine ganze Arbeit für ihre Beantwortung nötig ist.
    \item in einem Satz formuliert werden und nicht mehrere Fragen enthalten.
    \item offen gestellt werden, sodass sie nicht mit "Ja" oder "Nein" beantwortet werden kann.
\end{itemize}

